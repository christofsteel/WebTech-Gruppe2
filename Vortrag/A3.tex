\section{Aufgabe 2.3: Tooltip-Fenster}
\begin{frame}[<+->]{Aufgabe 2.3}
\normalsize
\begin{itemize}
\item Erstellen einer Tooltipbibliothek.
\item Tooltips bewegen sich mit der Maus.
\item Tooltips erscheinen erst nach einer bestimmten Anzahl an Millisekunden.
\item Die Bibliothek lässt sich ohne großen Aufwand in bestehende HTML-Dokumente einbinden.
\end{itemize}
\end{frame}
\begin{frame}[fragile]{tooltiptest.html}
\tiny{\begin{lstlisting}[language = HTML,
                                   mathescape = true, 
                   breaklines=true, 
  		   keywordstyle=\color{blue}\bfseries,
  stringstyle=\color{red}\ttfamily,
                   numbers = left, 
                   numbersep = 3pt]
<!DOCTYPE HTML PUBLIC "-//W3C//DTD HTML 4.01 Transitional//EN" "http://www.w3.org/TR/html4/loose.dtd">
<html>
    <head>
        <meta http-equiv="Content-Type" content="text/html; charset=utf-8">
        <title>Tooltip Beispiel</title>
        <script type="text/javascript" src="jstooltip.js"> </script>
        <script type="text/javascript">
                addTooltip('name', 'message', 1000);
                addTooltip('email', 'message2', 1000);
        </script>
    </head>
    
    <body>
		<form action="" method="get">
			<fieldset>
            	<legend>Eingaben</legend>
                <label>Name:</label>&nbsp;<input id="name" type="text"><br>
                <label>E-Mail:</label>&nbsp;<input id="email" type="text">
                <input id="abschicken" type="submit" value="Abschicken">
			</fieldset>
		</form>    	
    </body>
</html>
\end{lstlisting}
}
\end{frame}
\normalsize
\begin{frame}[<+->][fragile]{tooltiptest.html - Skriptsektion}
\tiny{\begin{lstlisting}[language = HTML,
  		   keywordstyle=\color{blue}\bfseries,
  stringstyle=\color{red}\ttfamily,
			numbers=left,
			firstnumber=6,
			numbersep = 3pt]
        <script type="text/javascript" src="jstooltip.js"> </script>
        <script type="text/javascript">
                addTooltip('name', 'message', 1000);
                addTooltip('email', 'message2', 1000);
        </script>
\end{lstlisting}}
\pause
\begin{itemize}
\normalsize
\item Zeile 6 bindet die Bibliothek ein
\item Zeile 8 und 9 zeigen, wie man HTML Elementen einen Tooltip hinzufügt.
\end{itemize}
\end{frame}
\begin{frame}[<+->]{Funktionen der Bibliothek}
\begin{itemize}
\item Funktionen der Bibliothek
\begin{description}
\small
\item[addTooltip] Wrapperfunktion, die \textbf{addTooltip\_} aufruft, sobald das Dokument geladen hat.
\item[addTooltip\_] Fügt unsichtbar dem Dokument alle Tooltips hinzu und erweitert die Elemente, die einen Tooltip bekommen um \textbf{onmouseover}, \textbf{onmousemove} und \textbf{onmouseout} Events.
\item[showTooltip] Positioniert den Tooltip und startet einen Timer, um ihn sichtbar zu machen.
\item[hideTooltip] Stoppt den Timer und versteckt den Tooltip wieder.
\item[moveTooltip] Positioniert den Tooltip anhand der Maus neu.
\end{description}
\normalsize
\item Globales Array
\small
\begin{description}
\item[timeout]Enthält die laufenden Timer, um bei Verlassen eines Elementes den Tooltiptimer abzubrechen.
\end{description}
\end{itemize}
\end{frame}
\begin{frame}[<+->][fragile]{jstooltip.js - addTooltip}
\tiny{\begin{lstlisting}[language=JavaScript, 
		   numbers=left,
		   numbersep=3pt,
		   breaklines=true]		 
timeout = new Array();

function addTooltip(Id, tooltip, milliseconds) {
        document.addEventListener('DOMContentLoaded', function(e){
                addTooltip_(Id, tooltip, milliseconds);
        }, false);
}
\end{lstlisting}}
\normalsize
\begin{description}
\item[Problem]Man kann nicht auf auf HTML Elemente zugreifen, wenn sie noch nicht geladen sind.
\item[Lösung]Wir warten, bis das Dokument vollständig geparst ist
\end{description}
\end{frame}
\begin{frame}[<+->][fragile]{jstooltip.js - addTooltip\_ Teil I}
\tiny{\begin{lstlisting}[language=JavaScript, 
		   numbers=left,
		   numbersep=3pt,
		   breaklines=true,
		   firstnumber=8]
function addTooltip_(Id, tooltip, milliseconds) {
        var element=document.getElementById(Id);
        tooltipId = "tooltip"+Id;
\end{lstlisting}}
\normalsize
\pause
\begin{itemize}
\item Zugriff auf das Element, welches den Tooltip bekommt.
\item Aus der ID wird eine TooltipId generiert.
\end{itemize}
\end{frame}
\begin{frame}[<+->][fragile]{jstooltip.js - addTooltip\_ Teil II}
\tiny{\begin{lstlisting}[language=JavaScript, 
		   numbers=left,
		   numbersep=3pt,
		   breaklines=true,
		   firstnumber=11]
        var tooltipdiv = document.createElement("div");
        tooltipdiv.id=tooltipId;
        tooltipdiv.innerHTML=tooltip;
        tooltipdiv.style.position="absolute";
        tooltipdiv.style.backgroundColor="#ffffff";
        tooltipdiv.style.border="1px solid #72B0E6";
        tooltipdiv.style.font="12px Verdana";
        tooltipdiv.style.color="#1A80DB";
        tooltipdiv.style.display="none";
        document.body.appendChild(tooltipdiv);
\end{lstlisting}}
\normalsize
\pause
\begin{itemize}
\item Ein div-Element wird für den Tooltip generiert, versteckt und dem Dokument hinzugefügt.
\end{itemize}
\end{frame}
\begin{frame}[<+->][fragile]{jstooltip.js - addTooltip\_ Teil III}
\tiny{\begin{lstlisting}[language=JavaScript, 
		   numbers=left,
		   numbersep=3pt,
		   breaklines=true,
		   firstnumber=21]
        element.setAttribute("onmouseover", "showTooltip(event,'"+tooltipId+"', '"+tooltip+"',"+milliseconds+");");
        element.setAttribute("onmousemove", "moveTooltip(event,'"+tooltipId+"');");
        element.setAttribute("onmouseout", "hideTooltip('"+tooltipId+"');");
}
\end{lstlisting}}
\normalsize
\pause
\begin{itemize}
\item Dem Element, welches den Tooltip bekommt, werden nun die entsprechenden Events hinzugefügt.
\item Für den Firefox wichtig ist das übergeben des \textbf{event}.
\end{itemize}
\end{frame}
\begin{frame}[<+->][fragile]{jstooltip.js - showTooltip}
\tiny{\begin{lstlisting}[language=JavaScript, 
		   numbers=left,
		   numbersep=3pt,
		   breaklines=true,
		   firstnumber=25]		 
function showTooltip(e,tooltipId, tooltip, milliseconds) {
        var e = e? e : window.event;
        tooltipdiv = document.getElementById(tooltipId);
        tooltipdiv.style.left=(e.pageX+10)+"px";
        tooltipdiv.style.top=(e.pageY+10)+"px";
        timeout[tooltipId]=window.setTimeout("tooltipdiv.style.display=\"block\"", milliseconds);
}		   
\end{lstlisting}}
\pause
\normalsize
\begin{itemize}
\item Browserabhängig wird auf \textbf{window.event} zugegriffen.
\item Das div wird abhängig von der aktuellen Mausposition gesetzt.
\item Nach \textit{milliseconds} Millisekunden wird das div angezeigt. Der Timer wird im Array \textit{timeout} gespeichert.
\end{itemize}

\end{frame}
\begin{frame}[<+->][fragile]{jstooltip.js - hideTooltip}
\tiny{\begin{lstlisting}[language=JavaScript, 
		   numbers=left,
		   numbersep=3pt,
		   breaklines=true,
		   firstnumber=32]		 
function hideTooltip(tooltipId) {
        window.clearTimeout(timeout[tooltipId]);
        tooltipdiv = document.getElementById(tooltipId);
        tooltipdiv.style.display="none";
}
\end{lstlisting}}
\pause
\normalsize
\begin{itemize}
\item Der Timer wird gestoppt und der Tooltip wird wieder versteckt.
\end{itemize}
\end{frame}
\begin{frame}[<+->][fragile]{jstooltip.js - moveTooltip}
\tiny{\begin{lstlisting}[language=JavaScript, 
		   numbers=left,
		   numbersep=3pt,
		   breaklines=true,
		   firstnumber=37]		 
function moveTooltip(e,tooltipId) {
        var e = e? e : window.event;
        tooltipdiv = document.getElementById(tooltipId);
        tooltipdiv.style.left=(e.pageX+10)+"px";
        tooltipdiv.style.top=(e.pageY+10)+"px";
}
\end{lstlisting}}
\normalsize
\pause
\begin{itemize}
\item Siehe \textbf{showTooltip}.
\end{itemize}
\end{frame}
