\section{Aufgabe 2.4: Tabellenzeilen in JavaScript dynamisch hervorheben}
\begin{frame}[<+->][fragile]{Aufgabe 2.4}
\tiny{ \begin{lstlisting}[language=JavaScript, 
		   numbers=left,
		   numbersep=3pt,
		   firstnumber= 7,
		   breaklines=true]
	function color(row) {
		row.oldcolor=row.getAttribute("bgColor");
		row.setAttribute("bgColor", "#90FF90");
	}
	function backcolor(row) {
		row.setAttribute("bgColor", row.oldcolor);
	}
</script>
\end{lstlisting}}	
\tiny{ \begin{lstlisting}[language = HTML,
                                   mathescape = true,  
                   numbers = left,
	        firstnumber=19 , 
                   numbersep = 3pt]
<tr bgcolor="#0099CC" onmouseover="color(this);" onmouseout="backcolor(this)">
  <th>Name:</th>
  <th>Telefon:</th>
  <th>Gebaeude/ Raum:</th>
</tr>
\end{lstlisting}}
\normalsize

\end{frame}