\section{Aufgabe 2.4: Tabellenzeilen in JavaScript dynamisch hervorheben}
\begin{frame}[<+->]{Aufgabe 2.4}
\begin{itemize}
\item Erweiterung der Datei \texttt{aufgabe2\_4.html} um Tabellenelemente hervorzuheben
\item Beim \textbf{mouseover} soll eine bestimmte Farbe als Hintergrund gesetzt werden
\item Beim \textbf{mouseout} soll die Ursprungsfarbe wieder hergestellt werden.
\end{itemize}
\end{frame}
\begin{frame}[<+->][fragile]{aufgabe2\_4.html Teil I}
\tiny{\begin{lstlisting}[language = HTML,
  		   keywordstyle=\color{blue}\bfseries,
  stringstyle=\color{red}\ttfamily,
                   breaklines=true, 
                   numbers = left, 
                   numbersep = 3pt]
<!DOCTYPE HTML PUBLIC "-//W3C//DTD HTML 4.01 Transitional//EN" "http://www.w3.org/TR/html4/loose.dtd">
<html>
<head>
<meta http-equiv="Content-Type" content="text/html; charset=utf-8">
<title>Hervorheben von Tabellenzeilen mit JavaScript</title>
<script type="text/javascript">
	function color(row) {
		row.oldcolor=row.getAttribute("bgColor");
		row.setAttribute("bgColor", "#90FF90");
	}
	function backcolor(row) {
		row.setAttribute("bgColor", row.oldcolor);
	}
</script>
</head>
\end{lstlisting}}
\normalsize
\pause
\begin{itemize}
\item In Zeile 8 wird die aktuelle Farbe einfach im \textit{row}-Objekt gespeichert.
\item und in Zeile 12 wieder verwendet.
\end{itemize}
\end{frame}
\begin{frame}[fragile]{aufgabe2\_4.html Teil II}
\tiny{\begin{lstlisting}[language = HTML,
  		   keywordstyle=\color{blue}\bfseries,
  stringstyle=\color{red}\ttfamily,
                   breaklines=true, 
		   firstnumber=16,
                   numbers = left, 
                   numbersep = 3pt]
<body>
	<table border="1">
    	<tr bgcolor="#0099CC" onmouseover="color(this);" onmouseout="backcolor(this)">
        	<th>Name:</th>
        	<th>Telefon:</th>
        	<th>Gebaeude/ Raum:</th>
        </tr>
\end{lstlisting}
\[\vdots\]
\begin{lstlisting}[language = HTML,
  		   keywordstyle=\color{blue}\bfseries,
  stringstyle=\color{red}\ttfamily,
                   breaklines=true, 
		   firstnumber=34,
                   numbers = left, 
                   numbersep = 3pt]
        <tr bgcolor="#CCCCCC" onmouseover="color(this);" onmouseout="backcolor(this)">
          <td>Jannach, Dietmar</td>
          <td>0231 755-7272</td>
          <td>GB IV, R338</td>
        </tr>
    </table>
</body>
</html>
\end{lstlisting}}
\normalsize
\pause
\begin{itemize}
\item Die Zeile wird einfach durch \textbf{this} an die Funktionen weitergegeben.
%\item Magic
\end{itemize}
\end{frame}
